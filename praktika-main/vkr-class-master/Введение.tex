\section*{ВВЕДЕНИЕ}
\addcontentsline{toc}{section}{ВВЕДЕНИЕ}

Агентство недвижимости — это компания, которая предлагает своим клиентам услуги, связанные с продажей, покупкой и арендой различной недвижимости

В современном мире информация является одним из самых ценных ресурсов, и ее эффективное управление имеет ключевое значение для успешного функционирования различных бизнес-секторов. Агентства недвижимости, оперирующие на динамичном рынке, сталкиваются с необходимостью хранения, обработки и анализа больших объемов данных о недвижимости, клиентах и сделках. Создание базы данных для агентства недвижимости становится неотъемлемым шагом, позволяющим оптимизировать рабочие процессы, повысить уровень обслуживания клиентов и улучшить внутреннюю организацию.

Цель данной работы заключается в разработке программно-информационной системы управления агентством недвижимости, с помощью которой можно быстро и эффективно осуществлять необходимые операции. В ходе разработки данной программно-информационной системы необходимо создать базу данных, поэтому в работе также будут рассмотрены ключевые аспекты проектирования баз данных, такие как определение сущностей и их взаимосвязей, выбор подходящей модели данных и применение современных технологий для реализации проекта.

Актуальность темы разработки базы данных для агентства недвижимости обуславливается несколькими ключевыми факторами:

1. Рост рынка недвижимости: В последнее время наблюдается стабильный рост интереса к покупке, продаже и аренде жилья. Это создает потребность в эффективных системах управления информацией, которая будет доступна в режиме реального времени.

2. Увеличение объема данных: В условиях цифровизации объем данных, связанных с недвижимостью, постоянно растет. Базы данных позволяют систематизировать, хранить и обрабатывать эти данные, что делает работу агентства более эффективной.

3. Улучшение клиентского обслуживания: Современные клиенты ожидают быстрого и качественного обслуживания. Наличие удобной и функциональной базы данных позволяет агентствам быстро реагировать на запросы клиентов, предоставляя актуальную информацию о доступных объектах недвижимости и их характеристиках.

4. Конкуренция на рынке: Агентства недвижимости сталкиваются с растущей конкуренцией. Эффективное использование баз данных может стать важным конкурентным преимуществом, позволяя оптимизировать внутренние бизнес-процессы и улучшить маркетинговые стратегии.

5. Автоматизация процессов: База данных позволяет автоматизировать множество рутинных задач, таких как учет объектов, управление заявками, расчеты и аналитика. Это активно снижает затраты времени и ресурсов, повышая общую производительность работы.

6. Анализ рынка: Система управления базами данных предоставляет возможность анализа тенденций и динамики рынка недвижимости, что является важным аспектом стратегического планирования для агентств.

7. Интеграция с другими системами: Возможность интеграции базы данных с другими информационными системами (например, CRM-системами, веб-сайтами и порталами объявления) позволяет создавать единую экосистему, что упрощает работу сотрудников и улучшает взаимодействие с клиентами.

С учетом вышеуказанного, разработка базы данных для агентства недвижимости является не только актуальной, но и необходимой для успешного функционирования бизнеса в условиях быстро меняющегося рынка.

В основные функции данной программно-информационной системы входит: возможность просмотра, добавление, редактирование, удаление информации о недвижимости, владельце недвижимости, покупателе, а также информация о сотрудниках агентства недвижимости. Приложение и база данных позволят эффективно управлять всеми аспектами агентства, сокращая время обработки и повышая общую производительность.

Основными задачами при проектировании и разработке приложения и БД являются:

•	исследование предметной области;

•	проектирование базы данных;

•	создание базы данных;

•	заполнение базы данных информацией;

•	разработка интерфейса;

•	реализация приложения.

Таким образом, данная работа направлена на создание функциональной и эффективной программно-информационной системы, способствующей успешному развитию агентства недвижимости и улучшению качества предоставляемых услуг.

Организация и объем работы представлены следующим образом: отчет включает введение, четыре раздела основной части, заключение, список литературы и два приложения. Общий объем выпускной квалификационной работы составляет 86 страниц.

Во введении обозначена цель исследования, сформулированы задачи, определена структура и дано краткое описание содержания каждого раздела.

Первый раздел, посвященный описанию технической характеристики предметной области, содержит цели и задачи агентства недвижимости, а также анелиз бизнес-процессов.

Второй раздел, соответствующий стадии технического задания, содержит перечень требований к разрабатываемому приложению.
Третий раздел, отражающий стадию технического проектирования, демонстрирует проектные решения для приложения и отображает структуру базы данных.

Четвертый раздел включает в себя результаты тестирования разработанного приложения.

В заключении суммированы ключевые результаты, достигнутые в процессе разработки.

Приложение А содержит графические материалы, а Приложение Б – фрагменты исходного кода.

