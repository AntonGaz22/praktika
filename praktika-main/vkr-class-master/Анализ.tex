\section{Анализ предметной области}
\subsection{Описание предметной области}

Агентство недвижимости (АН) — это организация, предоставляющая посреднические услуги при совершении сделок с недвижимостью. Деятельность АН охватывает широкий спектр операций, включая:

•	Покупка и продажа недвижимости: поиск объектов, соответствующих требованиям клиентов, организация просмотров, переговоры о цене, оформление договоров купли-продажи.

•	Аренда (долгосрочная и краткосрочная): подбор объектов для арендаторов, поиск арендаторов для собственников, составление договоров аренды.

•	Управление недвижимостью (Property Management): обслуживание объектов недвижимости, контроль платежей, ремонт и техническое обслуживание.

•	Консультационные услуги: оценка стоимости недвижимости, юридическое сопровождение сделок, помощь в получении ипотеки и т.д.

Эффективная работа АН предполагает обработку и анализ большого объёма информации, относящейся к различным категориям:

•	Объекты недвижимости: квартиры, дома, земельные участки, коммерческая недвижимость (офисы, магазины, склады) и т.д. Информация об объектах включает адрес, характеристики (площадь, количество комнат, материалы стен, год постройки и т.д.), фотографии, цену, описание, юридические документы.

•	Клиенты: потенциальные покупатели, продавцы, арендаторы и арендодатели. Информация о клиентах включает контактные данные, предпочтения, требования к недвижимости, историю сделок.

•	Сотрудники: риелторы, менеджеры, юристы, оценщики и другие специалисты. Информация о сотрудниках включает контактные данные, специализацию, историю работы, комиссионные.

•	Сделки: договоры купли-продажи, аренды, оказания услуг. Информация о сделках включает сведения об объекте, клиентах, сотрудниках, дате заключения, цене, условиях оплаты, комиссии АН.

•	Рекламные кампании и источники: информация о размещенных объявлениях, каналах привлечения клиентов, результатах рекламных кампаний.


\subsection{Цели и задачи агентства недвижимости}

Основными целями АН являются:

•	Максимизация прибыли: за счет успешных сделок и оказания качественных услуг.

•	Удовлетворение потребностей клиентов: обеспечение оптимального подбора недвижимости и сопровождение сделок.

•	Повышение эффективности работы сотрудников: оптимизация рабочих процессов, сокращение временных затрат и повышение производительности.

•	Расширение клиентской базы: привлечение новых клиентов и удержание существующих.

•	Укрепление репутации: Предоставление надежных и профессиональных услуг.

Для достижения этих целей АН выполняет следующие задачи:

•	Поиск и привлечение клиентов: Использование различных каналов рекламы и маркетинга.

•	Поиск и оценка объектов недвижимости: анализ рынка, поиск подходящих объектов, оценка их стоимости.

•	Организация просмотров и переговоров: встречи с клиентами, организация показов объектов, ведение переговоров о цене и условиях сделки.

•	Юридическое сопровождение сделок: проверка юридической чистоты объектов, подготовка и оформление договоров, регистрация сделок.

•	Финансовый контроль: контроль платежей, ведение бухгалтерского учета, расчет комиссионных.

•	Анализ рынка: сбор и анализ данных о рынке недвижимости, прогнозирование тенденций.


\subsection{Особенности предметной области, влияющие на проектирование БД}

При проектировании базы данных для АН необходимо учитывать следующие особенности:

•	Большой объем данных: база данных должна быть способна обрабатывать большие объемы данных о недвижимости, клиентах и сделках.

•	Необходимость поиска и фильтрации данных: пользователям нужна возможность быстрого поиска объектов по различным критериям (цена, площадь, местоположение, количество комнат и т. д.), а также фильтрации данных по различным параметрам.

•	Необходимость формирования отчетов: база данных должна обеспечивать возможность формирования различных отчетов, необходимых для анализа деятельности АН (отчеты о сделках, комиссионных, продажах и т. д.).

•	Безопасность данных: необходимо обеспечить защиту конфиденциальной информации о клиентах и сотрудниках.

•	Масштабируемость: база данных должна быть масштабируемой, чтобы учитывать рост агентства и увеличение объема обрабатываемой информации.

•	Интеграция с другими системами (возможно): интеграция с сайтом АН, CRM-системами, системами учета и отчетности.


\subsection{Анализ бизнес-процессов}
	
На основе анализа неформального описания предметной области были сформулированы бизнес-правила:

•	У каждого объекта недвижимости должен быть владелец

•	У каждого владельца должен быть телефон для связи

•	Каждая сделка имеет определенный тип

•	При регистрации каждой сделки данные о продавце и покупателе обязательны

•	Каждый объект недвижимости, сотрудник, владелец, покупатель, сделка должны иметь уникальный код ( ID)

•	Один владелец может иметь несколько квартир в собственности

•	Необходимо корректно вводить данные во всех полях

Ограничения целостности для таблицы ОБЪЕКТ НЕДВИЖИМОСТИ

•	Код объекта недвижимости является уникальным для каждого объекта недвижимости, разрешены только цифры

•	Количество комнат, цена- данные строки могут содержать только значения в виде цифр

•	Недопустимы пустые значения во всех полях, кроме срока аренды

Ограничения целостности для таблицы ПОКУПАТЕЛЬ/АРЕНДАТОР

•	Код покупателя/арендатора является уникальным для каждого покупателя/арендатора, разрешены только цифры

•	Паспортные данные, контактные данные- данные строки могут содержать только значения в виде цифр

•	Фамилия, имя, отчество – строка символов, длиной до 50 символов. Может содержать только буквы русского алфавита

•	Недопустимы пустые значения во всех полях, кроме отчества клиента

Ограничения целостности для таблицы ПРОДАВЕЦ/АРЕНДОДАТЕЛЬ

Правила для контроля уникальности в ключевом поле и требования к типам данных и ограничения на допустимые значения данных во всех полях разрабатываются по аналогии с приведенными для таблицы ПОКУПАТЕЛЬ/АРЕНДОДАТЕЛЬ.

Ограничения целостности для таблицы ДОГОВОР АРЕНДЫ

•	Код договора аренды является уникальным для каждого договора, разрешены только цифры

•	Дата заключения договора- календарная дата

•	При заключении договора обязательно должны быть заполнены данные о арендодателе, арендаторе, сотруднике агентства, объекте недвижимости

•	Недопустимы пустые значения во всех полях

Ограничения целостности для таблицы ДОГОВОР ПРОДАЖИ

•	Правила для контроля уникальности в ключевом поле и требования к типам данных и ограничения на допустимые значения данных во всех полях разрабатываются по аналогии с приведенными для таблицы ДОГОВОР АРЕНДЫ


